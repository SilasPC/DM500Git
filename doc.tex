\documentclass[12pt, a4paper]{article}

\usepackage{enumitem}
\usepackage[utf8]{inputenc}
\usepackage{amsmath}
\usepackage{amssymb}
\usepackage{fancyhdr}
\usepackage{lastpage}
\usepackage{svg}
\usepackage{graphicx}

\begin{document}

\begin{titlepage}
\begin{centering}
\large Syddansk Universitet $|$ IMADA \\
\today \\
DM500 - Studieintroduktion til datalogi | 20. f\\

\vspace{4CM}

\huge{\bf \LaTeX{} og Git } \\
%\Large{\bf Part 1: RunSimulation}

\vspace{\fill}

\fontsize{14}{19.2} {
    \selectfont
    KIAN BANKE LARSEN \vspace{5pt} \\
    KIM HAAGEN MATHIESEN \vspace{5pt} \\
    SILAS POCKENDAHL \vspace{5pt} \\
    \quad
} \\

\vspace{\fill}

%\includesvg[width=.4\textwidth]{SDU.svg}\\\vspace{.55cm}

\end{centering}

\thispagestyle{empty}
\end{titlepage}

\pagestyle{fancy}
\setlength{\headheight}{15pt}
\setlength{\headsep}{15pt}
\fancyhf{}
\rhead{\today}
\lhead{ SDU, DM500}
\chead{Kian B. L., Kim H. M., Silas. P. C.}
\cfoot{\thepage\ ud af \pageref{LastPage}}


\section{Kim}

Givet universet U, mængden S, samt mængderne A og B: \\
    \begin{gather*}
    U = \{1, 2, 3, 4, ..., 15\} \\ 
        S = \{1, 2, 3, 4\}\\
        A = \{ 2n | n \in S \}\\
        B = \{ 3n+2 | n \in S \}
    \end{gather*}
         
\begin{enumerate}[label=\alph*)]
	\item {
        Bestem mængden A.\\
		Mængden A er givet ved:\\
        \begin{math}
            A = \{ 2, 4, 6, 8 \}\\
        \end{math}
            Eftersom:\\
        \begin{math}
            A = \{ 2n | n \in S \} = \{1*2, 2*2, 3*2, 4*2\} = \{2, 4, 6, 8\}
	    \end{math}
    }
 	\item {
        Bestem mængden B.\\
		Mængden B er givet ved:\\
        \begin{math}
            B = \{ 5, 8, 11, 14 \}\\
        \end{math}
            Eftersom:\\
        \begin{math}
            B = \{ 3n+2 | n \in S \} = \{1*3+2, 2*3+2, 3*3+2, 4*3+2\} = \{5, 8, 11, 14\}
	    \end{math}
    }
 	\item {
        Bestem mængden \begin{math} A \cap B \end{math}.\\
		Mængden er givet ved:\\
        \begin{math}
            A \cap B = \{8\}\\
        \end{math}
            Eftersom, dette er det eneste element A og B har tilfælles.\\
    }
    \item {
        Bestem mængden \begin{math} A \cup B \end{math}.\\
		Mængden er givet ved:\\
        \begin{math}
            A \cup B = \{2, 4, 5, 6, 8, 11, 14\}\\
        \end{math}
            Eftersom, dette er elementerne A og B indeholder forenet/tilsammen.\\
    }   
    \item {
        Bestem mængden \begin{math} A -  B \end{math}.\\
		Mængden er givet ved:\\
        \begin{math}
            A - B = \{2, 4, 6\}\\
        \end{math}
            Eftersom, dette er elementerne i A fratrukket de fælles elementer for A og B, her kun elementet 8.\\
    }
    \item {
            Bestem mængden \begin{math} \overline{A} \end{math}.\\
		Mængden er givet ved:\\
        \begin{math}
            \overline{A} = \{1, 3, 5, 7, 9, 10, 11, 12, 13, 14, 15\}\\
        \end{math}
            Eftersom, dette er elementerne i universet U fratrukket elementerne i A.\\
    }          
\end{enumerate}


\section{Kian}
\section{Silas}
Følgende $\mathbb{R} \rightarrow \mathbb{R}$ funktioner er givet:
\begin{align*}
	f(x) & = x^2+x+1 \\
	g(x) & = 2x -2
\end{align*}
\begin{enumerate}[label=\alph*)]
	\item {
		Da $f(0)=f(-1)=1$, er $f$ ikke injektiv og dermed ikke bijektiv.
	}
	\item {
		Da $f$ ikke er bijektiv, kan den ikke have en invers.
	}
	\item {
		$f+g=f(x)+g(x)=x^2+x+1+2x-2=x^2+3x-1$
	}
	\item {
		$g\circ f=g(f(x))=2(x^2+x+1)-2=2x^2+2x$
	}
\end{enumerate}

\end{document}
