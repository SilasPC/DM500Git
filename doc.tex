% Document encoding.
\documentclass[12pt, a4paper]{article}
\usepackage[utf8]{inputenc}
\usepackage[danish]{babel}
\usepackage{microtype}
\usepackage{icomma}
\usepackage{parskip}

% Enumeration, tcolorbox and math.
\usepackage{enumitem} 
\usepackage{amsmath}
\usepackage{amssymb}
\usepackage{tcolorbox}


% Header and lastpage config.
\usepackage{fancyhdr} 
\usepackage{lastpage}
\pagestyle{fancy}
\setlength{\headheight}{15pt}
\setlength{\headsep}{15pt}
\fancyhf{}
\rhead{\today}
\lhead{ SDU, DM500}
\chead{Kian B. L., Kim H. M., Silas. P. C.}
\cfoot{\thepage\ ud af \pageref{LastPage}}

% Figures and pictures.
\usepackage{svg} 
\usepackage{graphicx}

% links, references, \ref{...}, \eqref{...}...
\usepackage[hidelinks, linktoc=all]{hyperref} 
\hypersetup{
    colorlinks = true,
    linkcolor = blue,
    citecolor = blue,
    urlcolor = blue
}
\urlstyle{same}

% For including GitLog in bash-style
% Minted is based on python via pygmintize. Hence python needs to be installed.
% A way to deal with this is to use listings instead as this is based on TeX.
\usepackage{minted}
\setminted[]{breaklines, 
    breakafter=d,
    frame=lines,
    framesep=2mm,
    baselinestretch=1.2,
    fontsize=\footnotesize,
    linenos}

\begin{document}
\begin{titlepage}
\begin{centering}
\large Syddansk Universitet $|$ IMADA \\
\today \\
DM500 | 20. f\\

\vspace{4CM}

\huge{\bf \LaTeX{} \& Git } \\

\vspace{\fill}

\fontsize{14}{19.2} {
    \selectfont
    KIAN BANKE LARSEN \vspace{5pt} \\
    KIM HAAGEN MATHIESEN \vspace{5pt} \\
    SILAS POCKENDAHL \vspace{5pt} \\
    \quad
} \\

\vspace{\fill}

\includesvg[width=.4\textwidth]{SDU.svg}\\\vspace{.55cm} % Install Inkscape and run pdfLaTeX with ``-shell-escape'' option :)

\end{centering}

\thispagestyle{empty}
\end{titlepage}

\section{Kim}
Givet universet U, mængden S, samt mængderne A og B: \\
    \begin{gather*}
    U = \{1, 2, 3, 4, ..., 15\} \\ 
        S = \{1, 2, 3, 4\}\\
        A = \{ 2n | n \in S \}\\
        B = \{ 3n+2 | n \in S \}
    \end{gather*}
         
\begin{enumerate}[label=\alph*)]
	\item {
        Bestem mængden A.\\
		Mængden A er givet ved:\\
        \begin{math}
            A = \{ 2, 4, 6, 8 \}\\
        \end{math}
            Eftersom:\\
        \begin{math}
            A = \{ 2n | n \in S \} = \{1*2, 2*2, 3*2, 4*2\} = \{2, 4, 6, 8\}
	    \end{math}
    }
 	\item {
        Bestem mængden B.\\
		Mængden B er givet ved:\\
        \begin{math}
            B = \{ 5, 8, 11, 14 \}\\
        \end{math}
            Eftersom:\\
        \begin{math}
            B = \{ 3n+2 | n \in S \} = \{1*3+2, 2*3+2, 3*3+2, 4*3+2\} = \{5, 8, 11, 14\}
	    \end{math}
    }
 	\item {
        Bestem mængden \begin{math} A \cap B \end{math}.\\
		Mængden er givet ved:\\
        \begin{math}
            A \cap B = \{8\}\\
        \end{math}
            Eftersom, dette er det eneste element A og B har tilfælles.\\
    }
    \item {
        Bestem mængden \begin{math} A \cup B \end{math}.\\
		Mængden er givet ved:\\
        \begin{math}
            A \cup B = \{2, 4, 5, 6, 8, 11, 14\}\\
        \end{math}
            Eftersom, dette er elementerne A og B indeholder forenet/tilsammen.\\
    }   
    \item {
        Bestem mængden \begin{math} A -  B \end{math}.\\
		Mængden er givet ved:\\
        \begin{math}
            A - B = \{2, 4, 6\}\\
        \end{math}
            Eftersom, dette er elementerne i A fratrukket de fælles elementer for A og B, her kun elementet 8.\\
    }
    \item {
            Bestem mængden \begin{math} \overline{A} \end{math}.\\
		Mængden er givet ved:\\
        \begin{math}
            \overline{A} = \{1, 3, 5, 7, 9, 10, 11, 12, 13, 14, 15\}\\
        \end{math}
            Eftersom, dette er elementerne i universet U fratrukket elementerne i A.\\
    }          
\end{enumerate}

\newpage
\section{Kian}\vspace{-5pt}
\textit{Reeksamen februar 2015 opgave 2.}
\begin{tcolorbox}
a) Hvilke af følgende udsagn er sande?
\begin{align}
	&\forall x \in \mathbb{N} \! : \exists y \in \mathbb{N} \! : x < y		\label{eq:1}\\
	&\forall x \in \mathbb{N} \! : \exists !y \in \mathbb{N} \! : x < y		\label{eq:2}\\
	&\exists y \in \mathbb{N} \! : \forall x \in \mathbb{N} \! : x < y		\label{eq:3}
\end{align}
\end{tcolorbox}\vspace{-5pt}
I udsagn \eqref{eq:1} hævdes det at der for alle \(x\) tilhørende naturlige tal, skal eksistere et \(y\) tilhørende naturlige tal, hvorved det gælder at \(x\) er mindre end \(y\). Udsagnet er \textbf{sandt}, fordi uanset hvilket tal der vælges fra mængden af naturlige tal, vil det altid være muligt at finde et tal der er større -- skyldes at mængden af naturlige tal er tælleligt uendelig.

Udsagn \eqref{eq:2} hævder det samme som ovenstående, dog tilføjes det at der kun eksistere netop ét \(y\). Jævnfør argumentationen for udsagn \eqref{eq:1}'s sandhed, kan udsagn \eqref{eq:2} kun være \textbf{falskt}, da der vil være uendeligt mange tal der er større end \(x\).

Udsagn \eqref{eq:3} påstår at der eksistere et \(y\) tilhørende naturlige tal, således at alle \(x\) i mængden naturlige tal, medfører at \(x\) er mindre end \(y\). Dette udsagn er \textbf{falskt}, grundet at der ikke findes et største tal i en uendelig stor mængde.
\begin{tcolorbox}
b) Angiv negeringen af udsagn \eqref{eq:1} fra spørgsmål a).\\ Negerings-operatoren (\(\neg\)) må ikke indgå i dit udsagn.
\end{tcolorbox}\vspace{-5pt}
Negeringen af udsagn \eqref{eq:1} udledes på følgende måde:\vspace{-5pt}
\begin{align}
\intertext{Hele udsagnet negeres:}
	&\neg(\forall x \in \mathbb{N}\! : \exists y \in \mathbb{N}\! : x < y)\\
\intertext{I henhold til De Morgans love for kvantorer flyttes negeringen ind i parentesen:}
	&\neg\forall x \in \mathbb{N}\! : \exists y \in \mathbb{N}\! : x < y\\
	&\exists x \in \mathbb{N}\! : \neg\exists y \in \mathbb{N}\! : x < y\\
\intertext{Negeringen af sammenligningsoperatoren $<$ er:}
	&\exists x \in \mathbb{N}\! : \forall y \in \mathbb{N}\! : \neg(x < y)\\
	&\exists x \in \mathbb{N}\! : \forall y \in \mathbb{N}\! : x \geq y
\end{align}
Hermed er udsagnet negeret.
\newpage

\section{Silas}
Følgende $\mathbb{R} \rightarrow \mathbb{R}$ funktioner er givet:
\begin{align*}
	f(x) & = x^2+x+1 \\
	g(x) & = 2x -2
\end{align*}
\begin{enumerate}[label=\alph*)]
	\item {
		Da $f(0)=f(-1)=1$, er $f$ ikke injektiv og dermed ikke bijektiv.
	}
	\item {
		Da $f$ ikke er bijektiv, kan den ikke have en invers.
	}
	\item {
		$f+g=f(x)+g(x)=x^2+x+1+2x-2=x^2+3x-1$
	}
	\item {
		$g\circ f=g(f(x))=2(x^2+x+1)-2=2x^2+2x$
	}
\end{enumerate}

\newpage
\section{Git-log}
\inputminted{bash}{git-log.txt}
\end{document}
